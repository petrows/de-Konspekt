\documentclass[12pt,a4paper]{report}
%\documentclass[10pt,a5paper]{report}

\usepackage{polyglossia}
\setdefaultlanguage[spelling=new]{german}
\usepackage[german,russian]{translator}
\usepackage{pgfgantt}

\setmainfont[Ligatures=TeX]{Liberation Serif}
\newfontfamily\cyrillicfont{Liberation Serif}
\setmonofont{Liberation Mono}
\newfontfamily\cyrillicfonttt{Liberation Mono}

\usepackage{geometry} % Меняем поля страницы
\geometry{left=2cm}% левое поле
\geometry{right=1cm}% правое поле
\geometry{top=1cm}% верхнее поле
\geometry{bottom=1.5cm}% нижнее поле

\usepackage{graphicx,xcolor}
\usepackage{longtable}
\usepackage{makecell}
% \usepackage[colorlinks=true,linkcolor=blue,urlcolor=black,bookmarksopen=true]{hyperref}
\usepackage[open,openlevel=1]{bookmark}
\graphicspath{ {./img/} }

\title{Немецкий язык: конспект}
\date{2018}

\definecolor{userdarkgreen}{RGB}{00, 128, 00}

\newcommand{\term}[1]{\texttt{\textbf{#1}}}
\newcommand{\satzew}[1]{\underline{#1}}
\newcommand{\ubersatze}[1]{\textit{#1}}
\newcommand{\nom}{ {\color{black}{\textbf{N}}} }
\newcommand{\akk}{ {\color{red}{\textbf{A}}} }
\newcommand{\dat}{ {\color{blue}{\textbf{D}}} }
\newcommand{\gen}{ {\color{userdarkgreen}{\textbf{G}}} }
% das Verb:
\newcommand{\dverb}{ {\color{blue}{\textbf{V}}} }


\begin{document}

\tableofcontents

% ========================================================================================================
\chapter{Личные местоимения}

\begin{longtable}{ c l l l l l }
\caption{Единственное число} \label{tab:long}\\
		& Я 	& Ты 	& Он  	& Она	& Оно	\\
\nom 	& ich 	& du	& er	& sie	& es	\\
\akk 	& mich 	& dich	& ihn	& sie	& es	\\
\dat 	& mir 	& dir	& ihm	& ihr	& ihm	\\
\end{longtable}

\begin{longtable}{ c l l l l l }
\caption{Множественное число} \label{tab:long}\\
		& Мы 	& Вы 	& Вы (уваж.)	& Они	\\
\nom 	& wir 	& ihr	& Sie			& sie	\\
\akk 	& uns 	& euch	& Sie			& sie	\\
\dat 	& uns 	& euch	& Ihnen			& ihnen	\\
\end{longtable}

% ========================================================================================================
\chapter{Приставки}

\paragraph{Не отделяемые}
\begin{itemize}
 \item er
 \item ver
 \item be
 \item ge
 \item zer
 \item miss
\end{itemize}

% ========================================================================================================
\chapter{Предлоги}
\section{Время}
\begin{itemize}
 \item \term{vor} + \dat - <...> назад (два дня назад). Ich war in geshaft vor zwei Tagen.
 \item \term{vor} + \dat - на протяжении чего-то (в прошлом). Ich war in Urlaub vor zwei Wochen.
 \item \term{seit} + \dat - в течении (по настоящее)
 \item \term{für} + \akk - на (период)
 \item \term{nach} + \dat - после (чего-то)
 \item \term{bei} + \dat - во время чего-то
 \item \term{in} + \dat - через
 \item \term{uber} - через (для прошлого)
 \item \term{bis} - до
 \item \term{ab} + \dat - начинать с чего-то или когда-то, может указывать на будущее. ab morgen - с утра
\end{itemize}

\section{Принадлежность}
\begin{itemize}
\item \term{von} от чего-то, чей-то. Аналог английского by. \ubersatze{Das Brathähnchen von Frau Kopf}.
\end{itemize}

% ========================================================================================================
\chapter{Склонение прилагательных}

\begin{longtable}{ c l l l l }
\caption{der / die / das} \label{tab:long} \\
		& M 	& F 	& N 	& pl 	\\
\nom 	& -e 	& -e	&	-e	& -en	\\
\akk 	& -en 	& -e	&	-e	& -en	\\
\dat 	& -en 	& -en	&	-en	& -en	\\
\gen 	& -en 	& -en	&	-en	& -en	\\
\end{longtable}

По правилам первой таблицы также склоняются прилагательные с dieser, jeder, jener, alle, manche, solche, welche, derselbe, beide.

alle beide keine die diese meine = во множ числе всегда -en, иначе -e

\begin{longtable}{ c l l l l }
\caption{ein / kein} \label{tab:long} \\
		& M 			& F 		& N 		& pl 	\\
\nom 	& ein -er 		& eine -e	& ein -es	& -e	\\
\akk 	& einen -en 	& eine -e	& ein -es	& -e	\\
\dat 	& einem -en 	& einer -en	& einem -en	& -en	\\
\end{longtable}

По правилам второй таблицы также склоняются прилагательные с kein, dein.

\begin{longtable}{ c l l l l }
\caption{Без артикля} \label{tab:long} \\
		& M 	& F 	& N 	& pl 	\\
\nom 	& -er 	& -e	& -es	& -e	\\
\akk 	& -en 	& -e	& -es	& -e	\\
\dat 	& -em 	& -er	& -em	& -en	\\
\gen 	& -en 	& -en	& -en	& -en	\\
\end{longtable}

По правилам третьей таблицы также склоняются прилагательные с andere, einige, etliche, folgende, mehrere, verschiedene, viele, wenige.

После alle, beide, keine, die, diese, meine - мы ставим в мн числе -en к прилагательному.

\chapter{Падежи}

\section{Nominativ}

Нормальная форма в предложении. Именительный падёж. Что?

\section{Akkusativ}

Винительный падеж. Кого? Что? Куда?

Падеж указывающий:
\begin{itemize}
 \item \term{предмет действия} (вопрос ЧТО?)
 \item \term{направление действия} (вопрос КУДА?)
\end{itemize}

\paragraph{Примеры:}
\begin{itemize}
\item Ich kaufe den Komputer.
~\\ \ubersatze{Я покупаю (что?) компьютер.}
\end{itemize}

\subsection{Предлоги всегда с Akkusativ}
\begin{itemize}
\item  durch für ohne bis um gegen entlang 
\end{itemize}

\section{Dativ}

Дательный падеж. Указывает на действие, направленно в пользу како-то объекта или место действия.

Падеж указывающий:
\begin{itemize}
 \item \term{направление действия} (вопрос КОМУ?)
 \item \term{место действия} (вопрос ГДЕ?)
\end{itemize}

\paragraph{Примеры:}

\begin{itemize}
\item Ich kaufe den Komputer der Mutter.
~\\ \ubersatze{Я покупаю (что?) компьютер (кому?) матери.}
\end{itemize}

\subsection{Предлоги всегда с Dativ}

\begin{itemize}
\item mit, nach, aus,
\item zu, von, bei,
\item außen, mitgegen, gegenuber,
\end{itemize}

\section{Genetiv}

Принадлежность в немецком языке выражается (так же как и в английском) при помощи окончания -s: Peters Arbeit (работа Петера). Но Петер – имя. А вот как с другими словами: В немецком же в основном просто на вопрос чей? – wessen?
В мужском и среднем роде артикль меняется на des (определенный) или eines (неопределенный), а так же добавляется окончание -(е)s к существительному. При этом односложные, короткие, существительные предпочитают в Genitiv прибавлять более длинное окончание -es, а остальные прибавляют -s: des Kindes Hundes Buches, des Arbeiters.

\paragraph{Примеры:}

\begin{itemize}
\item Während der Sommerferien fahren wir nach Köln.
~\\ \ubersatze{Во время летних каникул мы поедем в Кельн.}
\item Wegen der Kälte bleibe ich zu Hause.
~\\ \ubersatze{Из-за холода я остаюсь дома.}
\item Er wohnt außerhalb der Stadt.
~\\ \ubersatze{Он живет за городом (вне города).}
\item Unweit ihres Hauses.
~\\ \ubersatze{Недалеко от её дома.}
\end{itemize}

\subsection{Предлоги всегда с Genetiv}

\begin{itemize}
\item während -- во время
\item wegen -- из-за, по причине
\item außerhalb -- вне, за
\item infolge -- вследствие чего-либо, в связи с чем-либо
\item trotz -- вопреки, несмотря, на, невзирая на
\item unweit -- недалеко от, поблизости от
\item statt (anstatt) - вместо
\end{itemize}

\subsection{Предлоги Akkusativ / Dativ}
\begin{itemize}
\item in an auf vor hinter unter über neben zwischen 
\end{itemize}



\newpage
\chapter{Сравнительная степерь прилагательных}

Прилагательные имеют сравнительные формы - больше / меньше / наибольший и т.п.

\section{Общие правила построения}

\begin{itemize}
\item Положительная степень
~\\ \ubersatze{Быстрый самолёт}
\item Сравнительная степень
~\\ \ubersatze{Дорога самолётом быстрее}
\item Превосходная степень: Атрибутивная
~\\ \ubersatze{Наибыстрейший самолёт} - то есть самый быстрый самолёт на свете
\item Превосходная степень: Предикативная
~\\ \ubersatze{Самолётом ехать быстрее всего} - то есть самый быстрый вариант (из предложенных)
\end{itemize}

\begin{longtable}{ l l l l }
\caption{Сравнительные степени простых форм} \label{tab:long}\\
\thead{Положительная} & \thead{Сравнительная} & \thead{Превосходная \\ (атрибутивно)} & \thead{Превосходная \\ (предикативно)} \\  \hline
schnell & schnell\term{er} & \term{der/die/das} schnell\term{ste} & \term{am} schnell\term{sten} \\
schön & schön\term{er} & \term{der/die/das} schön\term{ste} & \term{am} schön\term{sten} \\
klein & klein\term{er} & \term{der/die/das} klein\term{ste} & \term{am} klein\term{sten} \\
neu & neu\term{er} & \term{der/die/das} neu\term{ste} & \term{am} neu\term{sten} \\
faul & faul\term{er} & \term{der/die/das} faul\term{ste} & \term{am} faul\term{sten} \\
\end{longtable}

\paragraph{Умлаут ?}

Kälte, Jünge

a, o, u -> ä, ö, ü


\paragraph{Умлаут не получают прилагательные с дефтонгом \term{au}}

\begin{itemize}
\item laut
\item lauter
\item lauterste
\end{itemize}

\paragraph{Умлаут не получают прилагательные с суффиксом: \term{er}, \term{eg}}

\begin{itemize}
\item laut
\item lauter
\item lauterste
\end{itemize}


\section{Особые формы}

\begin{longtable}{ l l l l }
\caption{Сравнительные степени особых форм} \label{tab:long}\\
\thead{Положительная} & \thead{Сравнительная} & \thead{Превосходная \\ (атрибутивно)} & \thead{Превосходная \\ (предикативно)} \\  \hline
gut		& besser	& \term{der/die/das} beste & \term{am} besten \\
viel	& mehr 		& - & \term{am} meisten \\
bald	& eher 		& - & \term{am} ehesten \\
gern	& lieber 	& \term{der/die/das} liebste & \term{am} liebsten \\
nah		& näher 	& \term{der/die/das} nähste & \term{am} nähsten \\
hoch	& hoher 	& \term{der/die/das} höhste & \term{am} höchsten \\
\end{longtable}

\section{Примеры}

Друг говорит на немецком языке лучше, чем я.
Mein Freund spricht Deutsch besser als ich.

Моя мама старше, чем я, но моя бабушка самая старшая в нашей семье.
Meine Mutter ist älter als ich, aber meine Opa ist am ältesten in der Familie.

Дни в ноябре короче, чем в сентябре.
Tage sind kürzer in November als in Semptember.

Дети проводят больше времени в школе.
Die Kinder verbringen mehr Zein in der Schule.

Студенты делают сейчас меньше ошибок, чем раньше.
Die Studenten machen weniger Fehlere nun, als vorher.

Твой доклад был самым интересным.
Dein Vortrag war am interessantesten.

Зима -- самое холодное время года.
Winter ist die kälteste Jahreszeit.

Вы можете говорить медленнее? Я не все понимаю.
Konnen Sie langsamer sprechen? Ich verstehe nicht alles.

Я работаю охотнее дома, чем в офисе.
Ich arbeite lieber zu Hause als im Büro.

Скоро придет зима и станет холоднее.
Bald kommnt der Winter und es wird kälter sein.

Макс был самым молодым в нашей команде.
Max war am jüngesten in unserer Mannschaft.

Мой брат сильнее в математике, чем я.
Mein Bruder ist stärker im Matematik als ich.

Я не люблю мясо, я ем охотнее рыбу.
Fleisch gefält mir nicht, esse ich lieber Fisch.

Мы чаще думаем об отпуске.
Wir denken häufiger an einen Urlaub.
Wir denken öfte an einen Urlaub.

Моя комната уютней, потому что я сама ее обставила.
Mein Zimmer bequemer, weil ich selbst eingerichtet habe.

Этот поезд едет быстрее всех поездов.
Das ist der schnellste Zug.

\section{Превосходная степень «один из»}

\term{einer (eine, eines)} + \gen (plural).

\section{Примеры}

\paragraph{Примеры:}
\begin{itemize}
\item Sie ist \satzew{eine} \satzew{der schönsten Sängerinnen} in der Welt.
~\\ \ubersatze{Она одна из самых красивых певиц в мире.}
\item Das ist \satzew{eines} \satzew{der besondersten Bücher}, die ich gelesen habe.
~\\ \ubersatze{Это одна из самых необычных книг, которые я читал.}
\end{itemize}

\newpage
\chapter{Причинно-следственные связи}

\section{Причина: Weil}

Используется для указания причины предложения: \ubersatze{я не пошёл на работу, \term{так как} я заболел.}

\paragraph{Структура предложения:} ~\\
\begin{itemize}
\item Обычное предложение, \term{weil} + глагол в конце.
\end{itemize}

\section{Следствие: Deshalb, Dass}

Используется для указания следствия предложения: \ubersatze{я не пошёл на работу, \term{поэтому} дома.} или \ubersatze{я думаю, \term{что} сегодня не пойду на работу.}

\paragraph{Структура предложения:} ~\\
\begin{itemize}
\item Обычное предложение, \term{deshalb} + глагол + обычная форма.
\item Обычное предложение, \term{dass} + глагол + обычная форма.
\end{itemize}

\section{Временная связь: Wenn, Als}

Используется для связи по времени:

\term{Wenn} - связь событий в настоящем и повторяющиеся. \ubersatze{Когда мы видимся с друзьями, мы обычно пьём.}

\term{Als} - связь с событием в прошлом (которое точно не повторится). \ubersatze{Когда я был маленький, я ходил в школу.}

\paragraph{Структура предложения:} ~\\
\begin{itemize}
\item Обычное проедложение, Wenn / Als + глагол в конце.
\item Wenn / Als + глагол в конце, глагол на первом месте.
\end{itemize}

\paragraph{Примеры:}
\begin{itemize}
\item Ich \satzew{gehe} zur Straße, \satzew{wenn} ich \satzew{gehen möchte}.
~\\ \ubersatze{Я иду на улицу, когда я хочу ходить.}
\item Ich \satzew{habe} eine Dinge gefunden, \satzew{als} ich \satzew{gegangen bin}.
~\\ \ubersatze{Я нашёл хреновину, когда я ходил.}
\item \satzew{Wenn} ich möchte essen, \satzew{koche} ich gern.
~\\ \ubersatze{Если я хочу есть, готовлю я с удовольствием.}
\end{itemize}


\newpage
\chapter{Связующие в предложении - gegensatz}

\section{И / А: dagegen}

Используется для связи предложения: одним нужно одно, \term{а} другим другое.

\paragraph{Структура предложения:} ~\\
\begin{itemize}
\item Нормальный порядок, \term{dagegen} \dverb{} + предложение.
\end{itemize}

\paragraph{Примеры:}
\begin{itemize}
    \item Maximilian mag moderne Möbelstücke, \satzew{dagagen} bevorzügt Monika im classishen Stil enzurichten.
          ~\\ \ubersatze{Максимилиан любит современную мебель, \satzew{а} Моника предпочла бы обставить квартиру в классическом стиле.}
\end{itemize}

\section{В то время как: während}

Используется для связи предложения: \term{в то время как} один хочет одно, другой хочет другое.

\paragraph{Структура предложения:} ~\\
\begin{itemize}
\item \term{Während} + предложение \dverb{}, \dverb{} + предложение.
\end{itemize}

\paragraph{Примеры:}
\begin{itemize}
    \item \satzew{Während} Erik Design von elektronishen Geäte wichtig findet, liegt Erika mehr Wert auf Phlantsen, Blumen und schone Vasen.
          ~\\ \ubersatze{\satzew{В то время как} Эрик считает важным дизайн электронных устройств, Эрика придает больше значение растениям, цветам и красивым вазам.}
\end{itemize}

\section{В отличие от: gegensatz}

Используется для связи предложения: \term{в отличие от} один хочет одно, другой хочет другое.

\paragraph{Структура предложения:} ~\\
\begin{itemize}
\item \term{Im gegenzatz zu \dat} + существительное, Нормальный порядок.
\end{itemize}

\paragraph{Примеры:}
\begin{itemize}
    \item \satzew{Im gegenzatz zur} alten Wohnungen, Badzimern in modedrnen Wohnungen sind groß, und sind Küchen offen.
          ~\\ \ubersatze{\satzew{В отличие от} старых квартир, ванные комнаты в современных квартирах большие, а кухни открытые.}
\end{itemize}

\section{Тогда как: wohingegen}

Используется для связи предложения: один хочет одно, \term{тогда как}, другой хочет другое.

\paragraph{Структура предложения:} ~\\
\begin{itemize}
\item Нормальный порядок, \term{wohingegen} + предложение \dverb{}.
\end{itemize}

\paragraph{Примеры:}
\begin{itemize}
    \item Einge Arkitekten arbeiten mit klaren Linen und Formen, \satzew{wohingegen}, mann im Hausen von Friedensreich Hundertwasser nicht einfach gerade Linen finden kann.
          ~\\ \ubersatze{Некоторые архитекторы работают с четкими линиями и формами, \satzew{тогда как} в домах Friedensreich Hundertwasser трудно найти прямые линии.}
\end{itemize}


\newpage

% ========================================================================================================
\chapter{Глаголы + zu + um / ohne / damit}

Сущеуствует несколько способов соединения нескольих глаголов в предложении.

\paragraph{V + zu + V - два глагола в предложении без связующей}

Используется для соединения нескольких глаголов в предложении. \ubersatze{Я пытаюсь немецкий изучать}. Zu не ставится для глаголов движения (на первом месте)!

\paragraph{Структура v+zu:} ~\\
\begin{itemize}
\item Объект \term{глагол 1} остальная часть предложения + zu \term{глагол 2}.
\item Для отделяемых приставок: zu помещается после приставки: \term{abzumachen}.
\item Для неотделяемых приставок из приставок: zu помещается перед словом: \term{zu vermachen}, \term{zu machen}.
\end{itemize}

\paragraph{Примеры:}
\begin{itemize}
\item Ich \satzew{versuche} die Deutsche Sprache \satzew{zu lernen}.
~\\ \ubersatze{Я пытаюсь немецкий учить.}
\item Ich \satzew{habe} gestern \satzew{vergessen}, meine Hausafgabe \satzew{zu machen}.
~\\ \ubersatze{Я вчера забыл сделать домашнюю работу.}
\item Ich \satzew{fahre} heute ins Geschaft \satzew{einkaufen}.
~\\ \ubersatze{Я еду в магазин закупаться.}
\end{itemize}

% ========================================================================================================
\paragraph{V + um / damit + zu + V = что-бы}

Конструкция \term{um+zu} и союз \term{damit} на русский язык переводятся одинаково \ubersatze{для того, чтобы}. Однако, в немецком языке существует серьезная разница при построение предложений с \term{um + zu} zu и \term{damit}. Давайте сравним:

\begin{itemize}
\item Ich rufe meinen Freund, \satzew{um} ihn an den Termin \satzew{zu} erinnern.
~\\ \ubersatze{Я звоню своему другу, чтобы напомнить ему о встрече.}
\item Ich rufe meinen Freund, \satzew{damit} er mir über seinen Urlaub erzählt.
~\\ \ubersatze{Я звоню моему другу, чтобы он рассказал мне о своем отпуске. (глагол вконец и спрягается)}
\end{itemize}

В первом случае и в главном, и в придаточном предложение субьект один - ich. Во втором случае, в предложение два субъекта - ich (в главном предложение) и er (в придаточном). По этой причине употребляется союз \term{damit}, а не конструкция \term{um+zu}.

\paragraph{Примеры:}
\begin{itemize}
\item Ich gehe zu shule \satzew{um} neue kentnise zubekommen.
~\\ \ubersatze{Я иду в школу что бы получить новые знания.}
\end{itemize}

% ========================================================================================================
\paragraph{V + ohne + zu + V = без (помощи) чего-то}

\paragraph{Примеры:}
\begin{itemize}
\item Ich uberzetse die Worte, \satzew{ohne} wortebuch zubenutsen.
~\\ \ubersatze{Я перевожу слова, не используя переводчик.}
\end{itemize}

% ========================================================================================================
\paragraph{V + statt / anstatt + zu + V = вместо чего-то}

\paragraph{Примеры:}
\begin{itemize}
\item Ich uberzetse die Worte, \satzew{statt} wortebuch zubenutsen.
~\\ \ubersatze{Я перевожу слова, вместо использования переводчика.}
\end{itemize}

% ========================================================================================================
\chapter{Passiv}

Используется для выражений что с предметом или кем-то происходит действие. \ubersatze{Эта книга читается ребенку}.

Действие которое не закончено (уровень B1).
Действие которое закончено (уровень B2).

\section{Präsents (настоящее)}

Настоящее время.

\paragraph{Структура пассива:} ~\\
\begin{itemize}
\item \term{wird} -- ед.число + Partezip II.
\item \term{werden} -- мн.число + Partezip II.
\end{itemize}

\paragraph{Примеры:}
\begin{itemize}
\item Das Buch \satzew{wird} \satzew{gelesen}.
~\\ \ubersatze{Эта книга читается.}
\item Das Buch \satzew{wird} von den Kind \satzew{gelesen}.
~\\ \ubersatze{Эта книга читается ребенку.}
\item Die Bucher \satzew{werden} \satzew{gelesen}.
~\\ \ubersatze{Эти книги читаются.}
\end{itemize}

Die Einladungen müssen dringend verschikt werden.

1. Обыкновенный пассив (prazens): Книга читается. Das Buch wird gelesen
2. Прошедший письменный (Prateretum): Книга читалась. Das Buch wurde gelesen.
3. Прошедший устный (Perfekt): Книга читалась. Das Buch ist gelesen worden.
4. Модальный: Книга должна быть прочитана. Das Buch muss gelesen werden.

\section{Vegangenheit (прошедшее): Präteritun / Perfekt}

\paragraph{Praterutum: Монолог или книжный} ~\\
\begin{itemize}
\item \term{wurde} -- ед.число + Partezip II.
\item \term{wurden} -- мн.число + Partezip II.
\end{itemize}

\paragraph{Perfekt: Диалог} ~\\
\begin{itemize}
\item \term{ist} -- ед.число + Partezip II \term{worden}.
\item \term{sind} -- мн.число + Partezip II \term{worden}.
\end{itemize}

\paragraph{Passiv mit Modalverben} ~\\
\begin{itemize}
\item \term{muss} -- ед.число + Partezip II \term{werden}.
\item \term{mussen} -- мн.число + Partezip II \term{werden}.
\end{itemize}

\paragraph{Ausname: zustand Passiv} ~\\

Пассив состояния: Das Fenster ist geschlossen.

\begin{itemize}
\item Das Fenster \satzew{ist} geschlossen.
~\\ \ubersatze{Окно закрыто}
\item Das Fenster \satzew{war} geschlossen.
~\\ \ubersatze{Окно было закрыто}
\end{itemize}

% ========================================================================================================
\chapter{Reflexive verben}
Рефлексивыне глаголы - для которых действие направлено на сам источник действия. Аналог русского ``ться''. \ubersatze{Я умываюсь.}
\paragraph{Формы прошедшего и модального времени:}
\begin{itemize}
\item Sie bewegt \satzew{sich} oft.
~\\ \ubersatze{Она быстро двигается.}
\item Sie ist \satzew{sich} oft bewegt.
~\\ \ubersatze{Она быстро двигалась.}
\item Sie will \satzew{sich} oft bewegen.
~\\ \ubersatze{Она хочет быстро двигаться.}
\item Bewegt sie \satzew{sich} oft?
~\\ \ubersatze{Она быстро двигается?}
\item Wo bewegt sie \satzew{sich} oft?
~\\ \ubersatze{Где она быстро двигается?}
\item Ist sie \satzew{sich} oft bewegt?
~\\ \ubersatze{Быстро она двигалась?}
\item Will sie \satzew{sich} oft bewegen?
~\\ \ubersatze{Хочет ли она быстро двигаться?}
\item Man muss \satzew{sich} schnell bewegen.
~\\ \ubersatze{Надо быстро двигаться.}
\end{itemize}

\paragraph{Исключения:}
\begin{itemize}
\item Du  kannst dir das Auto nicht leisten.
~\\ \ubersatze{Используется Dativ вместо обычной формы - mir, dir, ...}
\end{itemize}

% ========================================================================================================
\chapter{Konjunktiv II}
Если бы да кабы. Нереальное действие.

\begin{itemize}
    \item \satzew{wäre wär(e)st wären} - было бы
    \item \satzew{hätte hättest hätten} - имел бы
    \item \satzew{würde würdest würden} - для остальных случаев
\end{itemize}

\paragraph{Примеры:}
\begin{itemize}
    \item Ich \satzew{hätte} gerne Urlaub im Sommer.
          ~\\ \ubersatze{Я бы охотно взял (имел) отпуск летом.}
    \item Ich \satzew{wäre} lieber an der See.
          ~\\ \ubersatze{Я бы лучше побывал на море.}
    \item Wir \satzew{würden} gern mal ins Konzert gehen.
          ~\\ \ubersatze{Мы бы с удовольствием пошли в кино.}
    \item An deiner Stelle \satzew{würde} ich Führerschein machen.
          ~\\ \ubersatze{На твоём месте сделал бы я права.}
\end{itemize}

\paragraph{Использование с Wenn:}

\begin{enumerate}
    \item Wenn – глагол würden  конце и спрягается, основной перед ним в инфинитиве,
    \item Вторая часть предложения – глагол würden вначале, а основной в конце инфинитив ( если есть глаголы hätten wären) то спрягаются они и всё))
\end{enumerate}

\paragraph{Примеры:}
\begin{itemize}
    \item Wenn ich ins Krankenhaus gehen würde,  wäre ich nicht krank
    \item Wenn ich dich nicht treffen würde, würde ich allein zur Party gehen
    \item Wenn es möglich wäre, (dann) würde ich weniger arbeiten und mehr das Leben genießen.
          ~\\ \ubersatze{Если бы это было возможно, (тогда) я бы меньше работал и больше наслаждался жизнью.}
\end{itemize}

\paragraph{Модальные глаголы}
\begin{itemize}
    \item Если бы я мог бы спасть меньше, я мог бы выполнить больше работы.
    \item Wenn ich wenig schlafen könnte, müsste ich mehr Arbeit erladigen.
\end{itemize}

% ========================================================================================================
\chapter{Partizip}
Прилагательное из глагола: готовящийся, готовый.

\section{Partizip I - текущее действие}

Выражет текущее действие.

Форма:
\begin{itemize}
\item глагол + \term{-d} + окончание как у прилагательного
\end{itemize}

Примеры:
\begin{itemize}
\item Ein \satzew{laufender} Junge.
\item Das \satzew{schreibende} Frau.
\item Die auf dem Meer \satzew{segelnden} Schiffe sind schön anzusehen.
\item Drei Sprachen \satzew{sprechende} Personen sind als Überzetzer sehr begehert.
\end{itemize}

\section{Partizip II - звершенное действие (состояние)}

Причастие прошедшего времени выражает завершенное действие.

Форма:
\begin{itemize}
\item глагол в форме Perkect + окончание как у прилагательного
\end{itemize}

Означает уже завершенное дйствие:
\begin{itemize}
\item Die \satzew{gewäschene} Wäsche trocknet im Garten.
~\\ \ubersatze{Помытое бельё сохнет в саду.}
\item Weißt du, wo man gebrauchte Autos kaufen kann?
~\\ \ubersatze{Знаешь ли ты, как можно купить ипользованные автомобили?}
\item Tiago und Marie freuen sich auf ihren \satzew{geplanten} Urlaub.
~\\ \ubersatze{Тиаго и Мария находятся в запалнированном отпуске.}
\item Das \satzew{gekochte} Wasser.
~\\ \ubersatze{Сваренная вода}
\item Endlich fand sie den \satzew{verlorenen} Schlüssel.
\item Ich baue den neu \satzew{gekauften} Schrank auf.
\item Ich teile den frisch \satzew{gebackenen} Kuchen.
\item Er fährt das \satzew{reparierte} Fahrrad.
\end{itemize}

% ========================================================================================================
\chapter{Непрямые вопросы}
Вопрос не напрямую, \ubersatze{Могу ли я узнать, как вы долго здесь будете?}

\begin{itemize}
 \item Вводная часть, \term{ob} + глагол в конце -- для вопросов не содержащих вопросительного слова.
 \item Вводная часть, \term{вопросительное слово} + глагол в конце -- для остальных случаев.
\end{itemize}

\paragraph{Примеры:}
\begin{itemize}
 \item Ich würde gern wissen, \satzew{ob} Sie noch ein Zimmer frei \satzew{haben}?
 ~\\ \ubersatze{Я хотел бы узнать, есть ли у вас ещё одна свободная комната?}
 \item Darf ich fragen, \satzew{wie lange} Sie denn bei uns \satzew{bleiben möchten}?
 ~\\ \ubersatze{Могу ли я спросить, как долго вы у нас хотите оставаться?}
\end{itemize}

\chapter{Глаголы управления}

Это глагол + специальный предлог который используется с ним в паре.

\paragraph{Построение фраз:}
\begin{itemize}
 \item \term{Wo + (r) + предлог управления} - для вопросов про неодушевленные предметы.
 ~\\ \ubersatze{Wovon traumst du?}
 ~\\ \ubersatze{Worüber sprechst du?}
 \item \term{Предлог управления + Wen/Wem} - для вопросов про людей.
 ~\\ \ubersatze{Von wem traumst du?}
 ~\\ \ubersatze{Über wen sprechst du?}
 \item \term{da + (r) + предлог управления} - для обозначения слова-существительного "об этом" что бы не повторять предмет второй раз.
 ~\\ \ubersatze{Ich spreche auch daruber}
\end{itemize}

\begin{longtable}{ l l l l }
\caption{Список глаголов} \label{tab:long} \\
\multicolumn{4}{l}{keine} \\ \hline
		 & anbieten & \dat \akk & предлагать кому-то что-то \\
		 & erledigen & \akk{} & Улаживать (выполнять) что-то \\
		 & empfehlen & \dat{} \akk{} & Рекомендовать кому-то что-то \\
		 & helfen & \dat{} bei \dat{} & Помогать кому-то, в чем-то \\
		 & vorschlagen & \dat \akk & предлагать кому-то что-то \\
		 
\multicolumn{4}{l}{\term{an}} \\ \hline
		 & denken & an \akk & Думать (о чём-то) \\
	sich & erinnern & an \akk & Вспоминать о чем-то \\
		 & teilniehmen & an \dat & Принимать участие в чем-то \\		 
	sich & melden & an \dat & Зарегестрироваться где-то \\
	sich & gewöhnen & an \dat & Привыкать (к чему-то) \\
	sich & zweifeln & an \dat & Сомневаться в чем-то \\
	
\multicolumn{4}{l}{\term{auf}} \\ \hline
		 & achten & auf \akk & Обращать внимание на что-то \\
	sich & freuen & auf \akk & Радоваться (настоящее, будущее) \\
		 &        & über \akk & Радоваться (прошедшее) \\
    sein & stolz  & auf \akk & Гордиться чем-то \\
		 & Lust haben & auf \akk & Иметь желание на что-то \\
	sich & verlassen & auf \akk & Полагаться на кого-то \\
		 & verzichten & auf \akk & Отказываться от чего-то \\
	sich & vorbereiten & auf \akk & Готовиться к чему-то \\
		 & hoffen & auf \akk & Надеятся на что-то \\
		 & warten & auf \akk & Ждать чего-то \\
	
\multicolumn{4}{l}{\term{aus}} \\ \hline
		 & bestehen & aus \dat & состоять из чего-то \\
		 & stammen & aus \dat & происходить из чего-то \\
	
\multicolumn{4}{l}{\term{über}} \\ \hline
	sich & ärgern & über \akk & Злиться на что-то \\
	sich & beschweren & über \akk & Жаловаться на что-то \\
		 & erzahlen & über \akk & Рассказывать о чем-то \\
		 &          & von \dat & Рассказывать о чем-то \\
	
\multicolumn{4}{l}{\term{für}} \\ \hline
		 & danken & \dat für \akk & Благодарить кому-то за что-то \\
	sich & entschuldigen & für \akk & Извиняться за что-то \\
	sich & entscheiden & für \akk & Решиться на что-то \\
	sich & interessiert & für \akk & Интересоваться чем-то \\
	sich & verschwenden & für \akk & Тратить на что-то (спускать) \\
	
\multicolumn{4}{l}{\term{mit}} \\ \hline
	sich & beschäftigen & mit \dat & Заниматься чем-то (деятельность) \\
	sich & streiten & mit \dat um \akk & Ссориться с кем-то по поводу чего-то \\
		 & sprechen & mit \dat über \akk & Говорить с кем-то о чём-то \\
	sich & treffen & mit \dat & Встречаться с кем-то \\
	sich & verabreden & mit \dat & Договориться с кем-то \\
		 & vereinbaren & mit \dat \akk & Назначить (встречу) с кем-то что-то \\
	sein & zufrieden & mit \dat & Быть довольным чем-то \\

\multicolumn{4}{l}{\term{um}} \\ \hline
		 & bitten & um \akk & Просить о чём-то \\
	sich & kümmern & um \akk & Заботиться о ком-то (чем-то) \\
		 & es geht & um \akk & Идёт речь о чем-то \\
		 
\multicolumn{4}{l}{\term{von}} \\ \hline
		 & träumen & von \dat & Мечтать о чем-то \\
		 & abhängen & von \dat & Зависеть от кого-то \\
	sich & verabschieden & von \dat & Попрощаться с кем-то \\
		 & erzahlen & von \dat & Рассказывать о чем-то (используется чаще) \\
	sich & unterscheiden & von \dat & Отличаться от чего-то \\

\multicolumn{4}{l}{\term{vor}} \\ \hline
		 & Angst haben & vor \dat & Иметь страх на что-то \\
	sich & fürchten & vor \dat & Бояться чего-то \\

\multicolumn{4}{l}{\term{in}} \\ \hline
	sich & auskennen & in \dat & Разбираться в чем-то \\
	sich & verlieben & in \akk & Влюбиться в кого-то \\
	sich & irren & in \dat & Ошибаться в чем-то \\

\multicolumn{4}{l}{\term{zu}} \\ \hline
		 & gratulierien & \dat zu \dat & Поздравить кого-то с чем-то \\
		 & einladen & zu \dat & Пригласить куда-то \\

\multicolumn{4}{l}{\term{nach}} \\ \hline
		 & fragen & nach \dat & Спрашивать о чем-то \\

\end{longtable}

\newpage

% ========================================================================================================
\chapter{Relativesatz}

Предложение, состоящую из двух частей связанных словом который/которая/которые.

\begin{itemize}
 \item Вводное предложение, \term{der/den/dem/...} + глагол в конце.
\end{itemize}

\paragraph{Артикль зависит от падежа:}
\begin{itemize}
 \item Kennst du Maria, \satzew{die} in Arbeit gestern war?
 ~\\ \ubersatze{Знаешь ли ты Марию, которая на работе вчера была?}
 \item Magst du den Komputer, \satzew{den} ich habe gekauft hat?
 ~\\ \ubersatze{Нравится ли тебе компьютер, который я купил?}
\end{itemize}

% ========================================================================================================
\chapter{Präteritum}

Простое (книжное) прошедшее время.

\section{Модальные глаголы}

\paragraph{Модальные глаголы используются всегда в этой форме!}

\begin{longtable}{ c l l l l l l }
\caption{Модальные глаголы} \label{tab:long} \\
				& können 	& dürfen 	& müssen 	& sollen 	& mögen 	& wissen 	\\

	ich 		& konnte 	& durfte 	& mußte 	& sollte 	& mochte 	& wußte 	\\
	du 			& konntest 	& durftest	& mußtest	& solltest	& mochtest	& wußtest 	\\
	er/sie/es 	& konnte 	& durfte 	& mußte		& sollte	& mochte	& wußte		\\
	wir 		& konnten 	& durften 	& mußten	& sollten	& mochten	& wußten	\\
	ihr			& konntet 	& durftet 	& mußtet	& solltet	& mochtet	& wußtet	\\
	sie/Sie		& konnten 	& durften 	& mußten	& sollten	& mochten	& wußten	\\

\end{longtable}

% ========================================================================================================
\chapter{Zweitelige konnenktoren}

\begin{itemize}
\item \satzew{entweder} + \satzew{oder} - или, или
~\\ \ubersatze{Wie gehen entweder ins Kino oder in Restorant.}
\item \satzew{weder} + \satzew{noch} - ни, ни
~\\ \ubersatze{Ich esse wieder fleisch noch fish.}
\item \satzew{sowohl} + \satzew{als auch} - как, так и
~\\ \ubersatze{Ich kaufe sowohl den roten Rock als auch Grünen.}
\item \satzew{nicht nur} + \satzew{sondern auch} - не только, но и
~\\ \ubersatze{In Karslruhe gibt es nicht nur Zoo, sondern auch Schloss.}
\item \satzew{so} + \satzew{wie} - ?
\item \satzew{je} + \satzew{desto} - ?
\item \satzew{zwar} + \satzew{aber} (прямой порядок слов) - хоть, но
~\\ \ubersatze{Zwar ist er klug, aber nicht doch merksam.}
~\\ \ubersatze{Zwar gibt es an den Schweizer Unis in der Regel keine Zulassungsbeschränkung, aber man muss an der Uni St. Gallen eine Aufnahmeprüfung bestehen.}
\item \satzew{Einerseits} + \satzew{Andererseits} -  одной стороны, с другой стороны
~\\ \ubersatze{Einerseits gehe ich nach Arbeit, andererseits werde ich nichts da machen.}
\end{itemize}

% ========================================================================================================
\chapter{Verbverbindungen (ohne zu)}

Некоторые соединения могут быть сделаны без \satzew{zu}:
\begin{itemize}
    \item gehen
    \item lernen
    \item bleiben
    \item sehen (infinitiv im Perfekt)
    \item hören (infinitiv im Perfekt)
\end{itemize}

\begin{longtable}{ l l l }
    \caption{Verbverbindungen (ohne zu)} \label{tab:long} \\
    Verb &
    Präsens &
    Perfekt \\
    gehen &
    Sie \satzew{gehen} öfter mal \satzew{joggen}. &
    Sie \satzew{sind} öfter mal \satzew{joggen gegangen}. \\
    lernen &
    Viele leute \satzew{lernen} Zumba \satzew{tanzen}. &
    Viele leute \satzew{haben} Zumba \satzew{tanzen gelernt}. \\
    bleiben &
    Ich \satzew{bleibe} sofort \satzew{stehen}. &
    Ich \satzew{bin} sofort \satzew{stehen geblieben}. \\
    sehen &
    Er \satzew{sieht} mich auf einem Bein \satzew{stehen}. &
    Er \satzew{hat} mich auf einem Bein \satzew{stehen sehen}. \\
    hören &
    Ich \satzew{höre} ihn laut \satzew{lachen}. &
    Ich \satzew{habe} ihn laut \satzew{lachen hören}. \\
\end{longtable}

% ========================================================================================================
\chapter{Разобрать}
\begin{verbatim}
Würden - переводится как ``бы''
ich würde gern wissen, ... -- я хотел бы узнать, ...

Vorbei gehen an Datit. - идти мимо чего-то.

Zu + Dativ всегда! + все из считалочки. Даже если вопрос "куда"?
Ich gehe zur Arbeit. Ich gehe zur Schule.
Ich gehe in die Arbeit. Ich gehe in die Schule.



durch - сквозь, через (в т.ч. в переносном смысле через, посредством, с помощью, - я нашел информацию через интернет).
Ich habe diese Wochnung durch Anzeige gefunden.

für - для, на (время) - на два дня. За - что ты получил за работу, заплатить за что-то.
Ich fahre nach Spanien für 2 Wochen. - Еду на две недели.

ohne - без.

bis - temporal: до (завтра), local: до (какого-то места - еду до Франкфурта).

um - local: около, вокруг, um ... herum (вокруг чего-то). temporal: в (время). Wir sitzen um den Tisch.

gegen - против, около (время). Ich komme gegen funf.

entlang - вдоль (часто стоит в конце предложения). Sie fahrt den fluss entlang.

Предлоги AKK или DAT:
in auf an vor hinter unter neben zwischen über

Ich stelle die Lampe auf (куда?) den Tisch.
Die Lampe steiht auf (где?) dem Tisch.

Ich lege die Katze auf (куда?) den Teppich.
Die Katze liegt auf (где?) dem Teppich.

Ich hänge das Bild an (куда?) die Wand.
Das Bild hängt an (где?) der Wand.

Ich stecke einen Schlussel in (куда?) die Tur.
Der Schlussel steckt in (где?) der Tur.

Конструкция "что бы"

um ... zu - что бы что-то
Я иду в магазин что бы купить продукты: Ich gehe in dem Supermarkt, um liebensmittel zukaufen.
Если 2 разных человека в предложения, то вместо zu - damit (порядок слов =weil)
Ich lerne Deutsch damit meine Eltern zufrieden sind.




\end{verbatim}

\end{document}

