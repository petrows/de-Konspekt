\chapter{Падежи}

\section{Nominativ}

Нормальная форма в предложении. Именительный падёж. Что?

\section{Akkusativ}

Винительный падеж. Кого? Что? Куда?

Падеж указывающий:
\begin{itemize}
 \item \term{предмет действия} (вопрос ЧТО?)
 \item \term{направление действия} (вопрос КУДА?)
\end{itemize}

\paragraph{Примеры:}
\begin{itemize}
\item Ich kaufe den Komputer.
~\\ \ubersatze{Я покупаю (что?) компьютер.}
\end{itemize}

\section{Dativ}

Дательный падеж. Указывает на действие, направленно в пользу како-то объекта или место действия.

Падеж указывающий:
\begin{itemize}
 \item \term{направление действия} (вопрос КОМУ?)
 \item \term{место действия} (вопрос ГДЕ?)
\end{itemize}

\paragraph{Примеры:}

\begin{itemize}
\item Ich kaufe den Komputer der Mutter.
~\\ \ubersatze{Я покупаю (что?) компьютер (кому?) матери.}
\end{itemize}

\subsection{Предлоги всегда с Dativ}

\begin{itemize}
\item mit, nach, aus,
\item zu, von, bei,
\item außen, mitgegen, gegenuber,
\end{itemize}

