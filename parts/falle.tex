\chapter{Падежи}

\section{Nominativ}

Нормальная форма в предложении. Именительный падёж. Что?

\section{Akkusativ}

Винительный падеж. Кого? Что? Куда?

Падеж указывающий:
\begin{itemize}
 \item \term{предмет действия} (вопрос ЧТО?)
 \item \term{направление действия} (вопрос КУДА?)
\end{itemize}

\paragraph{Примеры:}
\begin{itemize}
\item Ich kaufe den Komputer.
~\\ \ubersatze{Я покупаю (что?) компьютер.}
\end{itemize}

\subsection{Предлоги всегда с Akkusativ}
\begin{itemize}
\item  durch für ohne bis um gegen entlang 
\end{itemize}

\section{Dativ}

Дательный падеж. Указывает на действие, направленно в пользу како-то объекта или место действия.

Падеж указывающий:
\begin{itemize}
 \item \term{направление действия} (вопрос КОМУ?)
 \item \term{место действия} (вопрос ГДЕ?)
\end{itemize}

\paragraph{Примеры:}

\begin{itemize}
\item Ich kaufe den Komputer der Mutter.
~\\ \ubersatze{Я покупаю (что?) компьютер (кому?) матери.}
\end{itemize}

\subsection{Предлоги всегда с Dativ}

\begin{itemize}
\item mit, nach, aus,
\item zu, von, bei,
\item außen, mitgegen, gegenuber,
\end{itemize}

\section{Genetiv}

Принадлежность в немецком языке выражается (так же как и в английском) при помощи окончания -s: Peters Arbeit (работа Петера). Но Петер – имя. А вот как с другими словами: В немецком же в основном просто на вопрос чей? – wessen?
В мужском и среднем роде артикль меняется на des (определенный) или eines (неопределенный), а так же добавляется окончание -(е)s к существительному. При этом односложные, короткие, существительные предпочитают в Genitiv прибавлять более длинное окончание -es, а остальные прибавляют -s: des Kindes Hundes Buches, des Arbeiters.

\paragraph{Примеры:}

\begin{itemize}
\item Während der Sommerferien fahren wir nach Köln.
~\\ \ubersatze{Во время летних каникул мы поедем в Кельн.}
\item Wegen der Kälte bleibe ich zu Hause.
~\\ \ubersatze{Из-за холода я остаюсь дома.}
\item Er wohnt außerhalb der Stadt.
~\\ \ubersatze{Он живет за городом (вне города).}
\item Unweit ihres Hauses.
~\\ \ubersatze{Недалеко от её дома.}
\end{itemize}

\subsection{Предлоги всегда с Genetiv}

\begin{itemize}
\item während -- во время
\item wegen -- из-за, по причине
\item außerhalb -- вне, за
\item infolge -- вследствие чего-либо, в связи с чем-либо
\item trotz -- вопреки, несмотря, на, невзирая на
\item unweit -- недалеко от, поблизости от
\item statt (anstatt) - вместо
\end{itemize}

