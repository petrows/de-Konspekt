\chapter{Падежи}

\section{Nominativ}

Нормальная форма в предложении. Используется для подлежащих обычно.

\section{Akkusativ}

Падеж указывающий:
\begin{itemize}
 \item \term{предмет действия} (вопрос ЧТО?)
 \item \term{направление действия} (вопрос КУДА?)
\end{itemize}

\paragraph{Примеры:}
\begin{itemize}
\item Ich kaufe den Komputer.
~\\ \ubersatze{Я покупаю (что?) компьютер.}
\item Ich \satzew{habe} eine Dinge gefunden, \satzew{als} ich \satzew{gegangen bin}.
~\\ \ubersatze{Я нашёл хреновину, когда я ходил.}
\item \satzew{Wenn} ich möchte essen, \satzew{koche} ich gern.
~\\ \ubersatze{Если я хочу есть, готовлю я с удовольствием.}
\end{itemize}
