\chapter{Сравнительная степерь прилагательных}

Прилагательные имеют сравнительные формы - больше / меньше / наибольший и т.п.

\section{Общие правила построения}

\begin{itemize}
\item Положительная степень
~\\ \ubersatze{Быстрый самолёт}
\item Сравнительная степень
~\\ \ubersatze{Дорога самолётом быстрее}
\item Превосходная степень: Атрибутивная
~\\ \ubersatze{Наибыстрейший самолёт} - то есть самый быстрый самолёт на свете
\item Превосходная степень: Предикативная
~\\ \ubersatze{Самолётом ехать быстрее всего} - то есть самый быстрый вариант (из предложенных)
\end{itemize}

\begin{longtable}{ l l l l }
\caption{Сравнительные степени простых форм} \label{tab:long}\\
\thead{Положительная} & \thead{Сравнительная} & \thead{Превосходная \\ (атрибутивно)} & \thead{Превосходная \\ (предикативно)} \\  \hline
schnell & schnell\term{er} & \term{der/die/das} schnell\term{ste} & \term{am} schnell\term{sten} \\
schön & schön\term{er} & \term{der/die/das} schön\term{ste} & \term{am} schön\term{sten} \\
klein & klein\term{er} & \term{der/die/das} klein\term{ste} & \term{am} klein\term{sten} \\
neu & neu\term{er} & \term{der/die/das} neu\term{ste} & \term{am} neu\term{sten} \\
faul & faul\term{er} & \term{der/die/das} faul\term{ste} & \term{am} faul\term{sten} \\
\end{longtable}

\paragraph{Умлаут не получают прилагательные с дефтонгом \term{au}}

\begin{itemize}
\item laut
\item lauter
\item lauterste
\end{itemize}

\paragraph{Умлаут не получают прилагательные с суффиксом: \term{er}, \term{eg}}

\begin{itemize}
\item laut
\item lauter
\item lauterste
\end{itemize}

\section{Особые формы}

\begin{longtable}{ l l l l }
\caption{Сравнительные степени особых форм} \label{tab:long}\\
\thead{Положительная} & \thead{Сравнительная} & \thead{Превосходная \\ (атрибутивно)} & \thead{Превосходная \\ (предикативно)} \\  \hline
gut		& besser	& \term{der/die/das} beste & \term{am} besten \\
viel	& mehr 		& \term{der/die/das} ? & \term{am} meisten \\
bald	& eher 		& \term{der/die/das} ? & \term{am} ehesten \\
gern	& lieber 	& \term{der/die/das} ? & \term{am} liebsten \\
nah		& näher 	& \term{der/die/das} ? & \term{am} nähsten \\
hoch	& hoher 	& \term{der/die/das} ? & \term{am} höchsten \\
\end{longtable}
