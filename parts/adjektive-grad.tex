\chapter{Сравнительная степерь прилагательных}

Прилагательные имеют сравнительные формы - больше / меньше / наибольший и т.п.

\section{Общие правила построения}

\begin{itemize}
\item Положительная степень
~\\ \ubersatze{Быстрый самолёт}
\item Сравнительная степень
~\\ \ubersatze{Дорога самолётом быстрее}
\item Превосходная степень: Атрибутивная
~\\ \ubersatze{Наибыстрейший самолёт} - то есть самый быстрый самолёт на свете
\item Превосходная степень: Предикативная
~\\ \ubersatze{Самолётом ехать быстрее всего} - то есть самый быстрый вариант (из предложенных)
\end{itemize}

\begin{longtable}{ l l l l }
\caption{Сравнительные степени простых форм} \label{tab:long}\\
\thead{Положительная} & \thead{Сравнительная} & \thead{Превосходная \\ (атрибутивно)} & \thead{Превосходная \\ (предикативно)} \\  \hline
schnell & schnell\term{er} & \term{der/die/das} schnell\term{ste} & \term{am} schnell\term{sten} \\
schön & schön\term{er} & \term{der/die/das} schön\term{ste} & \term{am} schön\term{sten} \\
klein & klein\term{er} & \term{der/die/das} klein\term{ste} & \term{am} klein\term{sten} \\
neu & neu\term{er} & \term{der/die/das} neu\term{ste} & \term{am} neu\term{sten} \\
faul & faul\term{er} & \term{der/die/das} faul\term{ste} & \term{am} faul\term{sten} \\
\end{longtable}

\paragraph{Умлаут ?}

Kälte, Jünge

a, o, u -> ä, ö, ü


\paragraph{Умлаут не получают прилагательные с дефтонгом \term{au}}

\begin{itemize}
\item laut
\item lauter
\item lauterste
\end{itemize}

\paragraph{Умлаут не получают прилагательные с суффиксом: \term{er}, \term{eg}}

\begin{itemize}
\item laut
\item lauter
\item lauterste
\end{itemize}


\section{Особые формы}

\begin{longtable}{ l l l l }
\caption{Сравнительные степени особых форм} \label{tab:long}\\
\thead{Положительная} & \thead{Сравнительная} & \thead{Превосходная \\ (атрибутивно)} & \thead{Превосходная \\ (предикативно)} \\  \hline
gut		& besser	& \term{der/die/das} beste & \term{am} besten \\
viel	& mehr 		& - & \term{am} meisten \\
bald	& eher 		& - & \term{am} ehesten \\
gern	& lieber 	& \term{der/die/das} liebste & \term{am} liebsten \\
nah		& näher 	& \term{der/die/das} nähste & \term{am} nähsten \\
hoch	& hoher 	& \term{der/die/das} höhste & \term{am} höchsten \\
\end{longtable}

\section{Примеры}

Друг говорит на немецком языке лучше, чем я. 
Mein Freund spricht Deutsch besser als ich.

Моя мама старше, чем я, но моя бабушка самая старшая в нашей семье.
Meine Mutter ist älter als ich, aber meine Opa ist am ältesten in der Familie.

Дни в ноябре короче, чем в сентябре. 
Tage sind kürzer in November als in Semptember.

Дети проводят больше времени в школе. 
Die Kinder verbringen mehr Zein in der Schule.

Студенты делают сейчас меньше ошибок, чем раньше. 
Die Studenten machen weniger Fehlere nun, als vorher. 

Твой доклад был самым интересным (der Vortrag – доклад)
Dein Vortrag war am interessantesten.

Зима – самое холодное время года. (die Jahreszeit – время года)
Winter ist die kälteste Jahreszeit.

Вы можете говорить медленнее? Я не все понимаю.
Konnen Sie langsamer sprechen? Ich verstehe nicht alles.

Я работаю охотнее дома, чем в офисе.
Ich arbeite lieber zu Hause als im Büro.

Скоро придет зима и станет холоднее. (bald – скоро)
Bald kommnt der Winter und es wird kälter sein.

Макс был самым молодым в нашей команде. (die Mannschaft – команда)
Max war am jüngesten in unserer Mannschaft.

Мой брат сильнее в математике, чем я.
Mein Bruder ist stärker im Matematik als ich.

Я не люблю мясо, я ем охотнее рыбу.
Fleisch gefält mir nicht, esse ich lieber Fisch.

Мы чаще думаем об отпуске. 
Wir denken häufiger an einen Urlaub.
Wir denken öfte an einen Urlaub.

Моя комната уютней, потому что я сама ее обставила (einrichten – обставлять)
Mein Zimmer bequemer, weil ich selbst eingerichtet habe.

Этот поезд едет быстрее всех поездов.
Das ist der schnellste Zug.

