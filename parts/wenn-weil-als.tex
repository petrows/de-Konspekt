\chapter{Причинно-следственные связи}

\section{Причина: Weil}

Используется для указания причины предложения: \ubersatze{я не пошёл на работу, \term{так как} я заболел.}

\section{Временная связь: Wenn, Als}

Используется для связи по времени:

\term{Wenn} - связь событий в настоящем и повторяющиеся. \ubersatze{Когда мы видимся с друзьями, мы обычно пьём.}

\term{Als} - связь с событием в прошлом (которое точно не повторится). \ubersatze{Когда я был маленький, я ходил в школу.}

\paragraph{Структура предложения:} ~\\
\begin{itemize}
\item Обычное проедложение, Wenn / Als + глагол в конце.
\item Wenn / Als + глагол в конце, глагол на первом месте.
\end{itemize}

\paragraph{Примеры:}
\begin{itemize}
\item Ich \satzew{gehe} zur Straße, \satzew{wenn} ich \satzew{gehen möchte}.
~\\ \ubersatze{Я иду на улицу, когда я хочу ходить.}
\item Ich \satzew{habe} eine Dinge gefunden, \satzew{als} ich \satzew{gegangen bin}.
~\\ \ubersatze{Я нашёл хреновину, когда я ходил.}
\item \satzew{Wenn} ich möchte essen, \satzew{koche} ich gern.
~\\ \ubersatze{Если я хочу есть, готовлю я с удовольствием.}
\end{itemize}

