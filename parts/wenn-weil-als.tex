\chapter{Причинно-следственные связи}

\section{Причина: Weil}

Используется для указания причины предложения: \ubersatze{я не пошёл на работу, \term{так как} я заболел.}

\paragraph{Структура предложения:} ~\\
\begin{itemize}
\item Следйствие, \term{weil} + причина + \dverb{}.
\end{itemize}

\section{Причина: Denn}

Используется для указания причины предложения, когда в обоих частях используется один и тот же субъект: \ubersatze{ У Маркуса нет времени, так как он должен работать}

\paragraph{Структура предложения:} ~\\
\begin{itemize}
\item Следствие, \term{denn} + \dverb + причина.
\end{itemize}

\paragraph{Примеры:}
\begin{itemize}
\item Marcus hat keine Zeit, denn er muss arbeiten.
~\\ \ubersatze{У Маркуса нет времени, так как он должен работать.}
\end{itemize}

\section{Следствие: Deshalb/Darum/Deswegen, Dass}

Используется для указания следствия предложения: \ubersatze{я не пошёл на работу, \term{поэтому} дома.} или \ubersatze{я думаю, \term{что} сегодня не пойду на работу.}

\paragraph{Структура предложения:} ~\\
\begin{itemize}
\item Главное действие, \term{deshalb / darum / deswegen} + \dverb{} + следствие.
\item Главное действие, \term{dass} + следствие + \dverb{}.
\end{itemize}

\paragraph{Примеры:}
\begin{itemize}
    \item Ich \satzew{denke}, \satzew{dass} alles möglich ist.
          ~\\ \ubersatze{Я думаю, \satzew{что} возможно всё.}
\end{itemize}

\section{Следствие: Dadurch, dass}

Используется для указания следствия предложения, аналогичен Weil, порядок обратен - следствие является главной частью предложения. Конструкция аналогична \term{потому, что}.

\paragraph{Структура предложения:} ~\\
\begin{itemize}
    \item Следствие \term{dadurch}, \term{dass} + Причина + \dverb{}.
    \item \term{dadurch}, \term{dass} + Причина + \dverb{}, \dverb{} + Следствие.
\end{itemize}

\paragraph{Примеры:}
\begin{itemize}
\item Weiniger Menschen lernen diese Sprachen \satzew{dadurch}, \satzew{dass} einige Sprachen an Bedeutung velieren.
~\\ \ubersatze{Меньше людей учат эти языки потому, что они теряют смысл.}
\item \satzew{Dadurch}, \satzew{dass} Latin nach vie vor in den Schulen gelernt wird, nimmt es eine besonderes Stellung ein.
~\\ \ubersatze{Потому, что Латинский играет особоую роль, изучается он как и прежде в школах.}
\end{itemize}

\section{Временная связь: Wenn, Als}

Используется для связи по времени:

\term{Wenn} - связь событий в настоящем и повторяющиеся. \ubersatze{Когда мы видимся с друзьями, мы обычно пьём.}

\term{Als} - связь с событием в прошлом (которое точно не повторится). \ubersatze{Когда я был маленький, я ходил в школу.}

\paragraph{Структура предложения:} ~\\
\begin{itemize}
\item Обычное проедложение, \term{Wenn / Als} + прошлое + \dverb{}.
\item \term{Wenn / Als} + прошлое + \dverb{}, \dverb{} + дополнение.
\end{itemize}

\paragraph{Примеры:}
\begin{itemize}
\item Ich \satzew{gehe} zur Straße, \satzew{wenn} ich \satzew{gehen möchte}.
~\\ \ubersatze{Я иду на улицу, когда я хочу ходить.}
\item Ich \satzew{habe} eine Dinge gefunden, \satzew{als} ich \satzew{gegangen bin}.
~\\ \ubersatze{Я нашёл хреновину, когда я ходил.}
\item \satzew{Wenn} ich möchte essen, \satzew{koche} ich gern.
~\\ \ubersatze{Если я хочу есть, готовлю я с удовольствием.}
\end{itemize}

