\chapter{Глаголы управления}

Это глагол + специальный предлог который используется с ним в паре.

\paragraph{Построение фраз:}
\begin{itemize}
 \item \term{Wo + (r) + предлог управления} - для вопросов про неодушевленные предметы.
 ~\\ \ubersatze{Wovon traumst du?}
 ~\\ \ubersatze{Worüber sprechst du?}
 \item \term{Предлог управления + Wen/Wem} - для вопросов про людей.
 ~\\ \ubersatze{Von wem traumst du?}
 ~\\ \ubersatze{Über wen sprechst du?}
 \item \term{da + (r) + предлог управления} - для обозначения слова-существительного "об этом" что бы не повторять предмет второй раз.
 ~\\ \ubersatze{Ich spreche auch daruber}
\end{itemize}

\begin{longtable}{ l l l l }
\caption{Список глаголов} \label{tab:long} \\
\multicolumn{4}{l}{keine} \\ \hline
	sich & anpassen & \dat & приспосабливаться \\
		 & anbieten & \dat \akk & предлагать кому-то что-то \\
		 & erledigen & \akk{} & Улаживать (выполнять) что-то \\
		 & empfehlen & \dat{} \akk{} & Рекомендовать кому-то что-то \\
		 & helfen & \dat{} bei \dat{} & Помогать кому-то, в чем-то \\
		 & vorschlagen & \dat \akk & предлагать кому-то что-то \\
		 
\multicolumn{4}{l}{\term{an}} \\ \hline
		 & denken & an \akk & Думать (о чём-то) \\
	sich & erinnern & an \akk & Вспоминать о чем-то \\
		 & teilniehmen & an \dat & Принимать участие в чем-то \\ 
	sich & melden & an \dat & Зарегестрироваться где-то \\
	sich & gewöhnen & an \dat & Привыкать (к чему-то) \\
	sich & zweifeln & an \dat & Сомневаться в чем-то \\
	
\multicolumn{4}{l}{\term{auf}} \\ \hline
		 & achten & auf \akk & Обращать внимание на что-то \\
	sich & freuen & auf \akk & Радоваться (настоящее, будущее) \\
		 &        & über \akk & Радоваться (прошедшее) \\
    sein & stolz  & auf \akk & Гордиться чем-то \\
		 & Lust haben & auf \akk & Иметь желание на что-то \\
	sich & verlassen & auf \akk & Полагаться на кого-то \\
		 & verzichten & auf \akk & Отказываться от чего-то \\
	sich & vorbereiten & auf \akk & Готовиться к чему-то \\
		 & hoffen & auf \akk & Надеятся на что-то \\
		 & warten & auf \akk & Ждать чего-то \\
	
\multicolumn{4}{l}{\term{aus}} \\ \hline
		 & bestehen & aus \dat & состоять из чего-то \\
		 & stammen & aus \dat & происходить из чего-то \\
	
\multicolumn{4}{l}{\term{bei}} \\ \hline
	sich & beklagen & bei \dat  & Плакаться кому-то \\
		 &          & über \akk & Плакаться на что-то \\
	
\multicolumn{4}{l}{\term{über}} \\ \hline
	sich & ärgern & über \akk & Злиться на что-то \\
	sich & beschweren & über \akk & Жаловаться на что-то \\
		 & erzahlen & über \akk & Рассказывать о чем-то \\
		 &          & von \dat & Рассказывать о чем-то \\
		 & nachdenken & über \akk & Обдумывать что-то \\
	
\multicolumn{4}{l}{\term{für}} \\ \hline
		 & danken & \dat für \akk & Благодарить кому-то за что-то \\
	sich & entschuldigen & für \akk & Извиняться за что-то \\
	sich & entscheiden & für \akk & Решиться на что-то \\
	sich & interessiert & für \akk & Интересоваться чем-то \\
		 & sorgen & für \akk & Беспокоится \\
	sich & verschwenden & für \akk & Тратить на что-то (спускать) \\
	
\multicolumn{4}{l}{\term{mit}} \\ \hline
	sich & beschäftigen & mit \dat & Заниматься чем-то (деятельность) \\
	sich & streiten & mit \dat um \akk & Ссориться с кем-то по поводу чего-то \\
		 & sprechen & mit \dat über \akk & Говорить с кем-то о чём-то \\
		 & stossen & mit \dat & Столкнуться \\
	sich & treffen & mit \dat & Встречаться с кем-то \\
	sich & verabreden & mit \dat & Договориться с кем-то \\
		 & vereinbaren & mit \dat \akk & Назначить (встречу) с кем-то что-то \\
	sein & zufrieden & mit \dat & Быть довольным чем-то \\

\multicolumn{4}{l}{\term{um}} \\ \hline
		 & bitten & um \akk & Просить о чём-то \\
	sich & kümmern & um \akk & Заботиться о ком-то (чем-то) \\
	sich \dat & sorgen machen & um \akk & Беспокоиться о ком-то \\
	sich \dat & Mühe geben & um \akk & Прикладвать услилия (для чего-то) \\
		 & es geht & um \akk & Идёт речь о чем-то \\
		 
\multicolumn{4}{l}{\term{von}} \\ \hline
		 & träumen & von \dat & Мечтать о чем-то \\
		 & abhängen & von \dat & Зависеть от кого-то \\
	sich & verabschieden & von \dat & Попрощаться с кем-то \\
		 & erzahlen & von \dat & Рассказывать о чем-то (используется чаще) \\
	sich & unterscheiden & von \dat & Отличаться от чего-то \\

\multicolumn{4}{l}{\term{vor}} \\ \hline
		 & Angst haben & vor \dat & Иметь страх на что-то \\
	sich & fürchten & vor \dat & Бояться чего-то \\

\multicolumn{4}{l}{\term{in}} \\ \hline
	sich & auskennen & in \dat & Разбираться в чем-то \\
	sich & verlieben & in \akk & Влюбиться в кого-то \\
	sich & irren & in \dat & Ошибаться в чем-то \\

\multicolumn{4}{l}{\term{zu}} \\ \hline
		 & gratulierien & \dat zu \dat & Поздравить кого-то с чем-то \\
		 & gehoren & zu \dat & Принадлежать чему-то \\
		 & einladen & zu \dat & Пригласить куда-то \\

\multicolumn{4}{l}{\term{nach}} \\ \hline
		 & fragen & nach \dat & Спрашивать о чем-то \\
		 
\end{longtable}

\paragraph{Исключения sich \dat}

В данных глаголах в \dat всегда ставится глагол \term{sich}! Меняется только 

\begin{itemize}
\item mich -> mir
\item dich -> dir
\end{itemize}

Остальные предлоги как обычно!
\begin{longtable}{ l l l }
	sich \dat & kaufen & Покупать (кому: себе) \\
	sich \dat & Mühe geben & Прилагать усилия (кому: себе) \\
	sich \dat & wünschen & Желать (кому: себе) \\
\end{longtable}

\begin{itemize}
\item Ich kaufe mir ein PC.
~\\ \ubersatze{Я покупаю себе PC}
\item Ich wünsche mir das Geschenk.
~\\ \ubersatze{я желаю себе подарок}
\item Du wünschst dir das Geschenk.
~\\ \ubersatze{Ты желаешь себе подарок}
\end{itemize}
