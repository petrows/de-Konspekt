\documentclass[12pt]{report}

%\usepackage{fontspec}
\usepackage{polyglossia}
\setdefaultlanguage{russian}
\setmainfont[Ligatures=TeX]{Liberation Serif}
\newfontfamily\cyrillicfonttt{Liberation Serif}

\usepackage{geometry} % Меняем поля страницы
\geometry{left=2cm}% левое поле
\geometry{right=1cm}% правое поле
\geometry{top=1cm}% верхнее поле
\geometry{bottom=1cm}% нижнее поле

\usepackage{graphicx,xcolor}
\graphicspath{ {./img/} }

\title{Немецкий язык: конспект}
\date{2018}

\newcommand{\term}[1]{\texttt{\textbf{#1}}}
\newcommand{\satzew}[1]{\underline{#1}}
\newcommand{\ubersatze}[1]{\textit{#1}}
\newcommand{\akk}{ {\color{red}{\textbf{akk}}} }
\newcommand{\dat}{ {\color{blue}{\textbf{dat}}} }

 
\begin{document}

\tableofcontents

\chapter{Распределить}

Würden - переводится как ``бы''
ich würde gern wissen, ... -- я хотел бы узнать, ...

% ========================================================================================================
\chapter{Приставки}

\paragraph{Не отделяемые}
\begin{itemize}
 \item er
 \item ver
 \item be
 \item ge
 \item zer
 \item miss
\end{itemize}

% ========================================================================================================
\chapter{Предлоги}
\paragraph{Время}
\begin{itemize}
 \item \term{ab} - начинать с чего-то или когда-то, может указывать на будущее.
\end{itemize}

\paragraph{Принадлежность}
\begin{itemize}
\item \term{von} от чего-то, чей-то. Аналог английского by. \ubersatze{Das Brathähnchen von Frau Kopf}.
\end{itemize}

\chapter{Падежи}

\section{Nominativ}

Нормальная форма в предложении. Именительный падёж. Что?

\section{Akkusativ}

Винительный падеж. Кого? Что? Куда?

Падеж указывающий:
\begin{itemize}
 \item \term{предмет действия} (вопрос ЧТО?)
 \item \term{направление действия} (вопрос КУДА?)
\end{itemize}

\paragraph{Примеры:}
\begin{itemize}
\item Ich kaufe den Komputer.
~\\ \ubersatze{Я покупаю (что?) компьютер.}
\end{itemize}

\subsection{Предлоги всегда с Akkusativ}
\begin{itemize}
\item  durch für ohne bis um gegen entlang
\end{itemize}

\section{Dativ}

Дательный падеж. Указывает на действие, направленно в пользу како-то объекта или место действия.

Падеж указывающий:
\begin{itemize}
 \item \term{направление действия} (вопрос КОМУ?)
 \item \term{место действия} (вопрос ГДЕ?)
\end{itemize}

\paragraph{Примеры:}

\begin{itemize}
\item Ich kaufe den Komputer der Mutter.
~\\ \ubersatze{Я покупаю (что?) компьютер (кому?) матери.}
\end{itemize}

\subsection{Предлоги всегда с Dativ}

\begin{itemize}
\item mit, nach, aus,
\item zu, von, bei,
\item außen, mitgegen, gegenuber,
\end{itemize}

\section{Genetiv}

Принадлежность в немецком языке выражается (так же как и в английском) при помощи окончания \term{-s}: Peters Arbeit (работа Петера). Но Петер – имя. А вот как с другими словами: В немецком же в основном просто на вопрос чей? – wessen?
В мужском и среднем роде артикль меняется на \term{des} (определенный) или \term{eines} (неопределенный), а так же добавляется окончание - \term{(е)s} к существительному. При этом односложные, короткие, существительные предпочитают в Genitiv прибавлять более длинное окончание \term{-es}, а остальные прибавляют \term{-s}: \ubersatze{des Kindes Hundes Buches, des Arbeiters}.

\paragraph{Примеры:}

\begin{itemize}
\item Während der Sommerferien fahren wir nach Köln.
~\\ \ubersatze{Во время летних каникул мы поедем в Кельн.}
\item Wegen der Kälte bleibe ich zu Hause.
~\\ \ubersatze{Из-за холода я остаюсь дома.}
\item Er wohnt außerhalb der Stadt.
~\\ \ubersatze{Он живет за городом (вне города).}
\item Unweit ihres Hauses.
~\\ \ubersatze{Недалеко от её дома.}
\end{itemize}

\subsection{Предлоги всегда с Genetiv}

\begin{itemize}
\item während -- во время
\item wegen -- из-за, по причине
\item innerhalb -- внутри (здания), а так же в течении (времени: неделя, год, ...)
\item außerhalb -- вне, за
\item infolge -- вследствие чего-либо, в связи с чем-либо
\item trotz -- вопреки, несмотря, на, невзирая на
\item unweit -- недалеко от, поблизости от
\item statt (anstatt) - вместо
\end{itemize}

\section{Дополнительно}

\subsection{Предлоги Akkusativ / Dativ}
\begin{itemize}
\item in an auf vor hinter unter über neben zwischen
\end{itemize}



\newpage
\chapter{Причинно-следственные связи}

\section{Причина: Weil}

Используется для указания причины предложения: \ubersatze{я не пошёл на работу, \term{так как} я заболел.}

\paragraph{Структура предложения:} ~\\
\begin{itemize}
\item Следйствие, \term{weil} + причина + \dverb{}.
\end{itemize}

\section{Причина: Denn}

Используется для указания причины предложения, когда в обоих частях используется один и тот же субъект: \ubersatze{ У Маркуса нет времени, так как он должен работать}

\paragraph{Структура предложения:} ~\\
\begin{itemize}
\item Следствие, \term{denn} + \dverb + причина.
\end{itemize}

\paragraph{Примеры:}
\begin{itemize}
\item Marcus hat keine Zeit, denn er muss arbeiten.
~\\ \ubersatze{У Маркуса нет времени, так как он должен работать.}
\end{itemize}

\section{Следствие: Deshalb/Darum/Deswegen, Dass}

Используется для указания следствия предложения: \ubersatze{я не пошёл на работу, \term{поэтому} дома.} или \ubersatze{я думаю, \term{что} сегодня не пойду на работу.}

\paragraph{Структура предложения:} ~\\
\begin{itemize}
\item Главное действие, \term{deshalb / darum / deswegen} + \dverb{} + следствие.
\item Главное действие, \term{dass} + следствие + \dverb{}.
\end{itemize}

\paragraph{Примеры:}
\begin{itemize}
    \item Ich \satzew{denke}, \satzew{dass} alles möglich ist.
          ~\\ \ubersatze{Я думаю, \satzew{что} возможно всё.}
\end{itemize}

\section{Следствие: Dadurch, dass}

Используется для указания следствия предложения, аналогичен Weil, порядок обратен - следствие является главной частью предложения. Конструкция аналогична \term{потому, что}.

\paragraph{Структура предложения:} ~\\
\begin{itemize}
    \item Следствие \term{dadurch}, \term{dass} + Причина + \dverb{}.
    \item \term{dadurch}, \term{dass} + Причина + \dverb{}, \dverb{} + Следствие.
\end{itemize}

\paragraph{Примеры:}
\begin{itemize}
\item Weiniger Menschen lernen diese Sprachen \satzew{dadurch}, \satzew{dass} einige Sprachen an Bedeutung velieren.
~\\ \ubersatze{Меньше людей учат эти языки потому, что они теряют смысл.}
\item \satzew{Dadurch}, \satzew{dass} Latin nach vie vor in den Schulen gelernt wird, nimmt es eine besonderes Stellung ein.
~\\ \ubersatze{Потому, что Латинский играет особоую роль, изучается он как и прежде в школах.}
\end{itemize}

\section{Временная связь: Wenn, Als}

Используется для связи по времени:

\term{Wenn} - связь событий в настоящем и повторяющиеся. \ubersatze{Когда мы видимся с друзьями, мы обычно пьём.}

\term{Als} - связь с событием в прошлом (которое точно не повторится). \ubersatze{Когда я был маленький, я ходил в школу.}

\paragraph{Структура предложения:} ~\\
\begin{itemize}
\item Обычное проедложение, \term{Wenn / Als} + прошлое + \dverb{}.
\item \term{Wenn / Als} + прошлое + \dverb{}, \dverb{} + дополнение.
\end{itemize}

\paragraph{Примеры:}
\begin{itemize}
\item Ich \satzew{gehe} zur Straße, \satzew{wenn} ich \satzew{gehen möchte}.
~\\ \ubersatze{Я иду на улицу, когда я хочу ходить.}
\item Ich \satzew{habe} eine Dinge gefunden, \satzew{als} ich \satzew{gegangen bin}.
~\\ \ubersatze{Я нашёл хреновину, когда я ходил.}
\item \satzew{Wenn} ich möchte essen, \satzew{koche} ich gern.
~\\ \ubersatze{Если я хочу есть, готовлю я с удовольствием.}
\end{itemize}


\newpage

% ========================================================================================================
\chapter{V + zu}

Используется для соединения нескольких глаголов в предложении. \ubersatze{Я пытаюсь немецкий изучать}.

\paragraph{Структура v+zu:} ~\\
\begin{itemize}
\item Объект \term{глагол 1} остальная часть предложения + zu \term{глагол 2}.
\item Для отделяемых приставок: zu помещается после приставки: \term{abzumachen}.
\item Для неотделяемых приставок из приставок: zu помещается перед словом: \term{zu vermachen}, \term{zu machen}.
\end{itemize}

\paragraph{Примеры:}
\begin{itemize}
\item Ich \satzew{versuche} die Deutsche Sprache \satzew{zu lernen}.
~\\ \ubersatze{Я пытаюсь немецкий учить.}
\item Ich \satzew{habe} gestern \satzew{vergessen}, meine Hausafgabe \satzew{zu machen}.
~\\ \ubersatze{Я вчера забыл сделать домашнюю работу.}
\end{itemize}

% ========================================================================================================
\chapter{Passiv}

Используется для выражений что с предметом или кем-то происходит действие. \ubersatze{Эта книга читается ребенку}.

\paragraph{Структура пассива:} ~\\
\begin{itemize}
\item \term{wird} -- ед.число + Partezip II.
\item \term{werden} -- мн.число + Partezip II.
\end{itemize}

\paragraph{Примеры:}
\begin{itemize}
\item Das Buch \satzew{wird} \satzew{gelesen}.
~\\ \ubersatze{Эта книга читается.}
\item Das Buch \satzew{wird} von den Kind \satzew{gelesen}.
~\\ \ubersatze{Эта книга читается ребенку.}
\item Die Bucher \satzew{werden} \satzew{gelesen}.
~\\ \ubersatze{Эти книги читаются.}
\end{itemize}

% ========================================================================================================
\chapter{Reflexive verben}
Рефлексивыне глаголы - для которых действие направлено на сам источник действия. Аналог русского ``ться''. \ubersatze{Я умываюсь.}
\paragraph{Формы прошедшего и модального времени:}
\begin{itemize}
\item Sie bewegt \satzew{sich} oft.
~\\ \ubersatze{Она быстро двигается.}
\item Sie ist \satzew{sich} oft bewegt.
~\\ \ubersatze{Она быстро двигалась.}
\item Sie will \satzew{sich} oft bewegen.
~\\ \ubersatze{Она хочет быстро двигаться.}
\item Bewegt sie \satzew{sich} oft? 
~\\ \ubersatze{Она быстро двигается?}
\item Wo bewegt sie \satzew{sich} oft?
~\\ \ubersatze{Где она быстро двигается?}
\item Ist sie \satzew{sich} oft bewegt? 
~\\ \ubersatze{Быстро она двигалась?}
\item Will sie \satzew{sich} oft bewegen?
~\\ \ubersatze{Хочет ли она быстро двигаться?}
\item Man muss \satzew{sich} schnell bewegen.
~\\ \ubersatze{Надо быстро двигаться.}
\end{itemize}

\paragraph{Исключения:}
\begin{itemize}
\item Du  kannst dir das Auto nicht leisten.
~\\ \ubersatze{Используется Dativ вместо обычной формы - mir, dir, ...}
\end{itemize}

% ========================================================================================================
\chapter{Непрямые вопросы}
Вопрос не напрямую, \ubersatze{Могу ли я узнать, как вы долго здесь будете?}

\begin{itemize}
 \item Вводная часть, \term{ob} + глагол в конце -- для вопросов не содержащих вопросительного слова.
 \item Вводная часть, \term{вопросительное слово} + глагол в конце -- для остальных случаев.
\end{itemize}

\paragraph{Примеры:}
\begin{itemize}
 \item Ich würde gern wissen, \satzew{ob} Sie noch ein Zimmer frei \satzew{haben}? 
 ~\\ \ubersatze{Я хотел бы узнать, есть ли у вас ещё одна свободная комната?}
 \item Darf ich fragen, \satzew{wie lange} Sie denn bei uns \satzew{bleiben möchten}? 
 ~\\ \ubersatze{Могу ли я спросить, как долго вы у нас хотите оставаться?}
\end{itemize}

\chapter{Разобрать}
Vorbei gehen an Datit. - идти мимо чего-то.

Zu + Dativ всегда! + все из считалочки. Даже если вопрос "куда"? 
Ich gehe zur Arbeit. Ich gehe zur Schule.
Ich gehe in die Arbeit. Ich gehe in die Schule.

Предлоги ВСЕГДА AKK:
durch für ohne bis um gegen entlang 

durch - сквозь, через (в т.ч. в переносном смысле через, посредством, с помощью, - я нашел информацию через интернет).
Ich habe diese Wochnung durch Anzeige gefunden.

für - для, на (время) - на два дня. За - что ты получил за работу, заплатить за что-то.
Ich fahre nach Spanien für 2 Wochen. - Еду на две недели.

ohne - без.

bis - temporal: до (завтра), local: до (какого-то места - еду до Франкфурта).

um - local: около, вокруг, um ... herum (вокруг чего-то). temporal: в (время). Wir sitzen um den Tisch.

gegen - против, около (время). Ich komme gegen funf.

entlang - вдоль (часто стоит в конце предложения). Sie fahrt den fluss entlang.

Предлоги AKK или DAT:
in auf an vor hinter unter neben zwischen über

Ich stelle die Lampe auf (куда?) den Tisch.
Die Lampe steiht auf (где?) dem Tisch.

Ich lege die Katze auf (куда?) den Teppich.
Die Katze liegt auf (где?) dem Teppich.

Ich hänge das Bild an (куда?) die Wand.
Das Bild hängt an (где?) der Wand.

Ich stecke einen Schlussel in (куда?) die Tur.
Der Schlussel steckt in (где?) der Tur.

Глаголы управления

Worüber - uber - с гласной, поэтому добавляется "r" - "о чем". Wo + r + uber - глагол с управлением
Sprechen mit D - с кем-то
Sprechen über A - о чем-то

darüber - об этом, можно не говорить второй раз сущ, ставить это вместо. Ich spreche auch daruber.

Träumen von dat - мечтать о чем-то

Sein zufrieden mit dat - быть довольным чем-то

sich freuen auf akk - радоваться чему-то
denken an Akk - думать над чем-то
sich treffen mit dat - встречаться с кем-то

ärgern uber - злиться 

для вопросов про одушевл. - предлог на первое место:

Auf wen freust du dich?

\begin{tabular}{ l l l l }
\multicolumn{4}{l}{keine} \\ \hline
		 & empfehlen & \dat{} \akk{} & Рекомендовать кому-то что-то \\
		 & helfen & \dat{} bei \dat{} & Помогать кому-то, в чем-то \\
\multicolumn{4}{l}{\term{an}} \\ \hline
		 & denken & an \akk & Думать (о чём-то) \\
	sich & erinnern & an \akk & Вспоминать о чем-то \\
		 & teilniehmen & an \dat & Принимать участие в чем-то \\
	sich & gewöhnen & an \dat & Привыкать (к чему-то) \\
	sich & zweifeln & an \dat & Сомневаться в чем-то \\
\multicolumn{4}{l}{\term{auf}} \\ \hline
		 & achten & auf \akk & Обращать внимание на что-то \\
	sich & freuen & auf \akk & Радоваться (настоящее, будущее) \\
		 &        & über \akk & Радоваться (прошедшее) \\
		 & Lust haben & auf \akk & Иметь желание на что-то \\
	sich & verlassen & auf \akk & Полагаться на кого-то \\
	sich & verzichten & auf \akk & Отказываться от чего-то \\
	sich & vorbereiten & auf \akk & Готовиться к чему-то \\
		 & hoffen & auf \akk & Надеятся на что-то \\
		 & warten & auf \akk & Ждать чего-то \\
		 
\multicolumn{4}{l}{\term{aus}} \\ \hline
		 & bestehen & aus \dat & состоять из чего-то \\
		 & stammen & aus \dat & происходить из чего-то \\
\multicolumn{4}{l}{\term{über}} \\ \hline
	sich & ärgern & über \akk & Злиться на что-то \\
	sich & beschweren & über \akk & Жаловаться на что-то \\
		 & erzahlen & uber \akk & Рассказывать о чем-то \\
\multicolumn{4}{l}{\term{für}} \\ \hline
		 & danken & \dat für \akk & благодарить кому-то за что-то \\
	sich & entschuldigen & für \akk & извиняться за что-то \\
	sich & entscheiden & für \akk & решиться на что-то \\
	sich & interessiert & für \akk & Интересоваться чем-то \\
\end{tabular}

Bitten um akk - просить о чём-то
Sich Beschäftigen mit dat - заниматься чем-то (деятельность)
Angst haben vor dat - иметь страх на что-то
Sich kümmern um akk - заботиться о ком-то (чем-то)
Sich Streiten mit dat um akk - ссориться с кем-то по поводу чего-то
Sich verlieben in akk - влюбиться в кого-то
Zufrieden sein mit dat - быть довольным чем-то
Es geht um akk - идёт речь о чем-то

Fragen nach dat - спрашивать о чем-то

Träumen von dat - мечтать о чем-то
Abhängen von dat - зависеть от кого-то
Sich verabschieden von dat - попрощаться с кем-то
Gratulierien dat zu dat - поздравить кого-то с чем-то
Sich verabreden mit dat - договориться с кем-то
Sich fürchten vor dat - бояться чего-то

einladen zu dat - пригласить куда-то
Erzahlen von dat - рассказывать о чем-то (используется чаще)

Sich Irren in dat - ошибаться в чем-то

vor()schlagen an()bieten dat akk - предлагать кому-то что-то


Предлоги времени

vor (+ Dativ) - <...> назад (два дня назад). Ich war in geshaft vor zwei Tagen.
seit (+ Dativ) - в течении 
für (+ Akk) - на (период)
nach (+ Dativ) - после (чего-то)
bei (+ Dativ) - во время чего-то
in (+ Dativ) - через
uber - через (для прошлого)
bis - до 
ab + dat - с (утра) + будущее. ab morgen - с утра

Конструкция "что бы"

um ... zu - что бы что-то
Я иду в магазин что бы купить продукты: Ich gehe in dem Supermarkt, um liebensmittel zukaufen.
Если 2 разных человека в предложения, то вместо zu - damit (порядок слов =weil)
Ich lerne Deutsch damit meine Eltern zufrieden sind.

Предлоги местоположения

Куда - Wohin + Akk

In die Bäckerei in die Schule in den Salon in den Supermarkt ins Kino 
Auf den Spielplatz Parkplatz Markt (*platz + исключение Markt)
Ans Meer an den See Fluss (всё что про воду)
Nach Spanien Deutschland (страны)
Zum Bahnhof zum Flughafen (исключения)
Zum Arzt Friseur (все ремесленники)

Где - Wo + dat

In der Bäckerei in der Schule Im Salon im Supermarkt 
Auf dem Markt  Spielplatz
Am Meer See Fluss Strand
In Spanien in Deutschand
am Bahnhof am Flughafen
beim  Arzt Friseur

\chapter{Relativesatz}

Предложение, состоящую из двух частей связанных словом который/которая/которые. 

\begin{itemize}
 \item Вводное предложение, \term{der/den/dem/...} + глагол в конце.
\end{itemize}

\paragraph{Артикль зависит от падежа:}
\begin{itemize}
 \item Ich würde gern wissen, \satzew{ob} Sie noch ein Zimmer frei \satzew{haben}? 
 ~\\ \ubersatze{Я хотел бы узнать, есть ли у вас ещё одна свободная комната?}
 \item Darf ich fragen, \satzew{wie lange} Sie denn bei uns \satzew{bleiben möchten}? 
 ~\\ \ubersatze{Могу ли я спросить, как долго вы у нас хотите оставаться?}
\end{itemize}

\end{document} 

