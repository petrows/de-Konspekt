\documentclass[12pt]{report}

%\usepackage{fontspec}
\usepackage{polyglossia}
\setdefaultlanguage{russian}
\setmainfont[Ligatures=TeX]{Liberation Serif}
\newfontfamily\cyrillicfonttt{Liberation Serif}

\usepackage{geometry} % Меняем поля страницы
\geometry{left=2cm}% левое поле
\geometry{right=1cm}% правое поле
\geometry{top=1cm}% верхнее поле
\geometry{bottom=1cm}% нижнее поле

\usepackage{graphicx,xcolor}
\graphicspath{ {./img/} }

\title{Немецкий язык: конспект}
\date{2018}

\newcommand{\term}[1]{\texttt{\textbf{#1}}}
\newcommand{\satzew}[1]{\underline{#1}}
\newcommand{\ubersatze}[1]{\textit{#1}}
 
\begin{document}

\tableofcontents

% ========================================================================================================
\chapter{Приставки}

\paragraph{Не отделяемые}
\begin{itemize}
 \item er
 \item ver
 \item be
 \item ge
 \item zer
 \item miss
\end{itemize}

% ========================================================================================================
\chapter{Предлоги}
\paragraph{Время}
\begin{itemize}
 \item \term{ab} - начинать с чего-то или когда-то, может указывать на будущее.
\end{itemize}

\paragraph{Принадлежность}
\begin{itemize}
\item \term{von} от чего-то, чей-то. Аналог английского by. \ubersatze{Das Brathähnchen von Frau Kopf}.
\end{itemize}

\chapter{Падежи}

\section{Nominativ}

Нормальная форма в предложении. Именительный падёж. Что?

\section{Akkusativ}

Винительный падеж. Кого? Что? Куда?

Падеж указывающий:
\begin{itemize}
 \item \term{предмет действия} (вопрос ЧТО?)
 \item \term{направление действия} (вопрос КУДА?)
\end{itemize}

\paragraph{Примеры:}
\begin{itemize}
\item Ich kaufe den Komputer.
~\\ \ubersatze{Я покупаю (что?) компьютер.}
\end{itemize}

\subsection{Предлоги всегда с Akkusativ}
\begin{itemize}
\item  durch für ohne bis um gegen entlang 
\end{itemize}

\section{Dativ}

Дательный падеж. Указывает на действие, направленно в пользу како-то объекта или место действия.

Падеж указывающий:
\begin{itemize}
 \item \term{направление действия} (вопрос КОМУ?)
 \item \term{место действия} (вопрос ГДЕ?)
\end{itemize}

\paragraph{Примеры:}

\begin{itemize}
\item Ich kaufe den Komputer der Mutter.
~\\ \ubersatze{Я покупаю (что?) компьютер (кому?) матери.}
\end{itemize}

\subsection{Предлоги всегда с Dativ}

\begin{itemize}
\item mit, nach, aus,
\item zu, von, bei,
\item außen, mitgegen, gegenuber,
\end{itemize}

\section{Genetiv}

Принадлежность в немецком языке выражается (так же как и в английском) при помощи окончания -s: Peters Arbeit (работа Петера). Но Петер – имя. А вот как с другими словами: В немецком же в основном просто на вопрос чей? – wessen?
В мужском и среднем роде артикль меняется на des (определенный) или eines (неопределенный), а так же добавляется окончание -(е)s к существительному. При этом односложные, короткие, существительные предпочитают в Genitiv прибавлять более длинное окончание -es, а остальные прибавляют -s: des Kindes Hundes Buches, des Arbeiters.

\paragraph{Примеры:}

\begin{itemize}
\item Während der Sommerferien fahren wir nach Köln.
~\\ \ubersatze{Во время летних каникул мы поедем в Кельн.}
\item Wegen der Kälte bleibe ich zu Hause.
~\\ \ubersatze{Из-за холода я остаюсь дома.}
\item Er wohnt außerhalb der Stadt.
~\\ \ubersatze{Он живет за городом (вне города).}
\item Unweit ihres Hauses.
~\\ \ubersatze{Недалеко от её дома.}
\end{itemize}

\subsection{Предлоги всегда с Genetiv}

\begin{itemize}
\item während -- во время
\item wegen -- из-за, по причине
\item außerhalb -- вне, за
\item infolge -- вследствие чего-либо, в связи с чем-либо
\item trotz -- вопреки, несмотря, на, невзирая на
\item unweit -- недалеко от, поблизости от
\item statt (anstatt) - вместо
\end{itemize}

\subsection{Предлоги Akkusativ / Dativ}
\begin{itemize}
\item in an auf vor hinter unter über neben zwischen 
\end{itemize}



\newpage
\chapter{Причинно-следственные связи}

\section{Причина: Weil}

Используется для указания причины предложения: \ubersatze{я не пошёл на работу, \term{так как} я заболел.}

\paragraph{Структура предложения:} ~\\
\begin{itemize}
\item Обычное предложение, \term{weil} + глагол в конце.
\end{itemize}

\section{Следствие: Deshalb, Dass}

Используется для указания следствия предложения: \ubersatze{я не пошёл на работу, \term{поэтому} дома.} или \ubersatze{я думаю, \term{что} сегодня не пойду на работу.}

\paragraph{Структура предложения:} ~\\
\begin{itemize}
\item Обычное предложение, \term{deshalb} + глагол + обычная форма.
\item Обычное предложение, \term{dass} + глагол + обычная форма.
\end{itemize}

\section{Временная связь: Wenn, Als}

Используется для связи по времени:

\term{Wenn} - связь событий в настоящем и повторяющиеся. \ubersatze{Когда мы видимся с друзьями, мы обычно пьём.}

\term{Als} - связь с событием в прошлом (которое точно не повторится). \ubersatze{Когда я был маленький, я ходил в школу.}

\paragraph{Структура предложения:} ~\\
\begin{itemize}
\item Обычное проедложение, Wenn / Als + глагол в конце.
\item Wenn / Als + глагол в конце, глагол на первом месте.
\end{itemize}

\paragraph{Примеры:}
\begin{itemize}
\item Ich \satzew{gehe} zur Straße, \satzew{wenn} ich \satzew{gehen möchte}.
~\\ \ubersatze{Я иду на улицу, когда я хочу ходить.}
\item Ich \satzew{habe} eine Dinge gefunden, \satzew{als} ich \satzew{gegangen bin}.
~\\ \ubersatze{Я нашёл хреновину, когда я ходил.}
\item \satzew{Wenn} ich möchte essen, \satzew{koche} ich gern.
~\\ \ubersatze{Если я хочу есть, готовлю я с удовольствием.}
\end{itemize}


\newpage

% ========================================================================================================
\chapter{V + zu}

Используется для соединения нескольких глаголов в предложении. \ubersatze{Я пытаюсь немецкий изучать}.

\paragraph{Структура v+zu:} ~\\
\begin{itemize}
\item Объект \term{глагол 1} остальная часть предложения + zu \term{глагол 2}.
\item Для отделяемых приставок: zu помещается после приставки: \term{abzumachen}.
\item Для неотделяемых приставок из приставок: zu помещается перед словом: \term{zu vermachen}, \term{zumachen}.
\end{itemize}

\paragraph{Примеры:}
\begin{itemize}
\item Ich \satzew{versuche} die Deutsche Sprache \satzew{zulernen}.
~\\ \ubersatze{Я пытаюсь немецкий учить.}
\item Ich \satzew{habe} gestern \satzew{vergessen}, meine Hausafgabe \satzew{zumachen}.
~\\ \ubersatze{Я вчера забыл сделать домашнюю работу.}
\end{itemize}

% ========================================================================================================
\chapter{Passiv}

Используется для выражений что с предметом или кем-то происходит действие. \ubersatze{Эта книга читается ребенку}.

\paragraph{Структура пассива:} ~\\
\begin{itemize}
\item \term{wird} -- ед.число + Partezip II.
\item \term{werden} -- мн.число + Partezip II.
\end{itemize}

\paragraph{Примеры:}
\begin{itemize}
\item Das Buch \satzew{wird} \satzew{gelesen}.
~\\ \ubersatze{Эта книга читается.}
\item Das Buch \satzew{wird} von den Kind \satzew{gelesen}.
~\\ \ubersatze{Эта книга читается ребенку.}
\item Die Bucher \satzew{werden} \satzew{gelesen}.
~\\ \ubersatze{Эти книги читаются.}
\end{itemize}

% ========================================================================================================
\chapter{Reflexive verben}
Рефлексивыне глаголы - для которых действие направлено на сам источник действия. Аналог русского ``ться''. \ubersatze{Я умываюсь.}
\paragraph{Формы прошедшего и модального времени:}
\begin{itemize}
\item Sie bewegt \satzew{sich} oft.
~\\ \ubersatze{Она быстро двигается.}
\item Sie ist \satzew{sich} oft bewegt.
~\\ \ubersatze{Она быстро двигалась.}
\item Sie will \satzew{sich} oft bewegen.
~\\ \ubersatze{Она хочет быстро двигаться.}
\item Bewegt sie \satzew{sich} oft? 
~\\ \ubersatze{Она быстро двигается?}
\item Wo bewegt sie \satzew{sich} oft?
~\\ \ubersatze{Где она быстро двигается?}
\item Ist sie \satzew{sich} oft bewegt? 
~\\ \ubersatze{Быстро она двигалась?}
\item Will sie \satzew{sich} oft bewegen?
~\\ \ubersatze{Хочет ли она быстро двигаться?}
\item Man muss \satzew{sich} schnell bewegen.
~\\ \ubersatze{Надо быстро двигаться.}
\end{itemize}

% ========================================================================================================
\chapter{Непрямые вопросы}
Вопрос не напрямую, \ubersatze{Могу ли я узнать, как вы долго здесь будете?}

\begin{itemize}
 \item Вводная часть, \term{ob} + глагол в конце -- для вопросов не содержащих вопросительного слова.
 \item Вводная часть, \term{вопросительное слово} + глагол в конце -- для остальных случаев.
\end{itemize}

\paragraph{Примеры:}
\begin{itemize}
 \item Ich würde gern wissen, \satzew{ob} Sie noch ein Zimmer frei \satzew{haben}? 
 ~\\ \ubersatze{Я хотел бы узнать, есть ли у вас ещё одна свободная комната?}
 \item Darf ich fragen, \satzew{wie lange} Sie denn bei uns \satzew{bleiben möchten}? 
 ~\\ \ubersatze{Могу ли я спросить, как долго вы у нас хотите оставаться?}
\end{itemize}
 
\end{document} 

